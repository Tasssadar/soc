% Když budete cokoli psát: ukládejte starší verze vždy odděleně, abyste se 
% k nim mohli kdykoli vrátit. Kusy textu, které jste se rozhodli nepoužít, taky ukládejte do zvláštního souboru. Smazat se to dá vždycky, ale psát to znova je opruz. 

% A POŘÁD ZÁLOHUJTE. POŘÁD !!!

\documentclass[12pt, a4paper, oneside]{article} 
% velikost písma, stránky, typ dokumentu - detaily viz literatura

\usepackage{czech} % nastavení češtiny
%\usepackage[latin2]{inputenc}
%\usepackage[cp1250]{inputenc} % pro win1250
\usepackage[utf8]{inputenc}
\usepackage{wrapfig} % nastavení obtékání textu
\usepackage{graphicx,amsmath} % nastavení grafiky, matematiky
%\usepackage{subfig} % více obrázků vedle sebe 

\usepackage{tocloft} %přidá tečky do obsahu ke kapitolám /sekcím 
\renewcommand{\cftsecdotsep}{\cftdotsep}

\usepackage[bookmarksopen,colorlinks,plainpages=false,linkcolor=black,urlcolor=black,citecolor=black,filecolor=black,menucolor=black,unicode=true]{hyperref}
%bookmarksopen - open up bookmark tree 
%colorlinks - zbarví odkazy (implicitně orámovaný nezbarvený text)
%urlcolor - barva odkazů (implicitně magenta) 
%linkcolor=black - barva odkazů v obsahu (implicitně red)


% \usepackage{parskip} - zapne americké odstavce v celé práci

\addtolength{\textwidth}{-2mm} 
\addtolength{\hoffset}{4mm}  % posun textu kvůli kroužkové vazbě  

\setlength{\intextsep}{5mm} % nastavení mezery okolo obrázků

% nastavení příkazu >\figcaption pro popis čehokoli, jako by to byly obrázky 
\makeatletter   
\newcommand\figcaption{\def\@captype{figure}\caption}
\makeatother

\def\refname{Literatura} 
% přejmenuje anglický název Reference na české Literatura


%\makeindex % příprava pro výrobu indexu (jestli ho chcete)

%%    VLNKA <fileinput>  KkSsVvZzOoUuAaIi        
% Defaultni  koncovka pro <fileinput> je  ".tex"
%FIXME: haze error
%\cstieon % Vypne chovani vlnky jako tvrde mezery v matematickem rezimu

%%%%%%%%%%%%%%%%%%%%%%%%%%%%%%%%%%%%%%%%%%%%%%%%%%%%%%%%%%%%%%%
%V PROSTŘEDÍ ROVNIC SE NESMÍ VYSKYTOVAT PRÁZDNÝ ŘÁDEK
%
%PROGRAMY VLNKA A CSINDEX SE MUSÍ SPUSTIT SAMOSTATNĚ
%%%%%%%%%%%%%%%%%%%%%%%%%%%%%%%%%%%%%%%%%%%%%%%%%%%%%%%%%%%%%%%

% definice příkazů 
\newcommand{\D}{\medskip \noindent} % nový odstavec v "americkém" formátování 
\newcommand{\B}{\textbf} %tučné písmo
\newcommand{\A}{\mathbf} %tučné písmo v matematickém režimu
\newcommand{\TO}{\ensuremath{\boldsymbol\Omega}} % tučný znak velké omega - pro ohmy
\newcommand{\I}{\index}  % vytváří položku indexu (asi nepoužijete)
\newcommand{\Deg}[1][]{\ensuremath{{#1}^\circ}} % vysází značku stupně Celsia
\newcommand{\Def}{\footnotesize Definice: \normalsize}
\newcommand{\Pos}{\footnotesize Experiment: \normalsize}
\newcommand{\Odv}{\footnotesize Odvození: \normalsize}
\newcommand{\Vym}{\footnotesize Vymezení pojmu: \normalsize}
\newcommand{\Ob}{obrázek }
\newcommand{\It}{\textit}  % kurzíva
\newcommand{\M}{\mathrm}   % v prostředí rovnic nastaví normální písmo (místo kurzívy ) 
\newcommand{\F}{\footnotesize} % zmenšená velikost písma
\newcommand{\N}{\normalsize} % normální velikost písma
%\newcommand{\U}{\underline}  % podtržené písmo
\newcommand{\e}{\ensuremath} 
% další příkaz se aplikuje, pouze, když jste v matematickém režimu

%\hyphenation{Pusť-me pla-tí hod-no-ty do-sa-dí-me za-da-né}
% dělení slov, kdyby implicitní nevyhovovalo

\linespread{1.3} 
% řádkování 1,5x  
% použijete podle situace  

\unitlength=1mm % nastavení volby jednotek 

% konec hlavičky
%%%%%%%%%%%%%%%%%%%%%%%%%%%%%%%%%%%%%%%%%%%%%%%%%%%%%%%%%%%%%%%%%%%
%%%%%%%%%%%%%%%%%%%%%%%%%%%%%%%%%%%%%%%%%%%%%%%%%%%%%%%%%%%%%%%%%%%

\begin{document} % začátek textové části 

% titulní strana
\pagestyle{empty} % vynechá číslování
 
\voffset = -20mm % posun začátku textu výš
\enlargethispage{60mm} % zvětší oblast tisku pro tuto stránku   

\begin{center}
 
\Large \B{STŘEDOŠKOLSKÁ ODBORNÁ ČINNOST}

\vspace{60mm}

\huge %\LARGE
\B{GRAFICKÉ UŽIVATELSKÉ ROZHRANÍ} 
% na titulní straně může být stručnější, pokud je to potřeba  

\Large

\vspace{90mm}


\B{Vojtěch Boček} \\

\vspace{40mm}

\B{Brno 2011}


\end{center}

\newpage % konec titulní strany 
%%%%%%%%%%%%%%%%%%%%%%%%%%%%%%%%%%%%%%%%%%%%%%%%%%%%%%%%%%%%%%%%%%%%%%%%%%%

% vnitřní titulní strana
\voffset = -20mm % posun začátku textu výš
\enlargethispage{60mm} % zvětší oblast tisku pro tuto stránku   

\begin{center}

\Large \B{STŘEDOŠKOLSKÁ ODBORNÁ ČINNOST}  \\
\vspace{10mm}
 \normalsize 
\B{Obor SOČ: 18. Informatika}% číslo a název - vyplníme spolu 

\vspace{45mm}

\LARGE %\huge 
\B{GRAFICKÉ UŽIVATELSKÉ ROZHRANÍ} 
\end{center}  
\large

\vspace{50mm}


\begin{tabbing}
\hspace{10mm} \= \hspace{30mm}  \=   \kill % nastavení zarážek 
  \> \B{Autor:}  \> \B{Vojtěch Boček}        \\[8mm] 
  \> \B{Škola:}   \> \B{SPŠ a~VOŠ technická, }     \\
  \>              \> \B{Sokolská 1 602 00 Brno}    \\[8mm]

  \> \B{Konzultant:} \> \B {Jakub Streit} 
\end{tabbing}

\vspace{20mm}

\begin{center}
\B{Brno 2012}

\end{center}
\normalsize
%%%%%%%%%%%%%%%%%%%%%%%%%%%%%%%%%%%%%%%%%%%%%%%%%%%%%%%%%%%%%%%%%%%%%%%%%%%
\newpage  % Prohlášení o autorství  
\voffset = 0mm % posun začátku textu zpět

~ % musí to tu být, aby fungovala svislá mezera

\vspace{10mm}

\section*{Prohlášení}

Prohlašuji, že jsem svou práci vypracoval samostatně, použil jsem pouze 
podklady (literaturu, SW atd.) citované v~práci a~uvedené v~přiloženém seznamu 
a~postup při zpracování práce je v~souladu se zákonem č. 121/2000 Sb., o~právu 
autorském, o~právech souvisejících s~právem autorským a~o~změně některých 
zákonů (autorský zákon) v~platném znění. 
 
\vspace{20mm} 
 
\noindent V~Brně  dne: 6.3.2012 \hspace{50mm}                 podpis:   
 

%%%%%%%%%%%%%%%%%%%%%%%%%%%%%%%%%%%%%%%%%%%%%%%%%%%%%%%%%%%%%%%%%%%%%%%%%%%
\newpage   % Poděkování - nepovinné 

~ % musí to tu být, aby fungovala svislá mezera

\vspace{140mm}

\section*{Poděkování}

 Děkuji Jakubu Streitovi za rady, obětavou pomoc, velkou trpělivost a~podnětné připomínky poskytované během práce na tomto projektu. % nebo cokoli dle Vašeho uvážení 
 
 Dále děkuji organizaci DDM Junior, za poskytnutí podpory.
 
 Také bych chtěl poděkovat panu profesorovi Mgr. Miroslavu Burdovi za všeobecnou pomoc s prací.  

\D Tato práce byla vypracována za finanční podpory JMK.
 

%%%%%%%%%%%%%%%%%%%%%%%%%%%%%%%%%%%%%%%%%%%%%%%%%%%%%%%%%%%%%%%%%%%%%%%%%%%
\newpage   % Anotace 
~ % musí to tu být, aby fungovala svislá mezera
\vspace{10mm}

\section*{Anotace }

    Cílem této práce bylo vytvořit uživatelské prostředí určené k parsování a~zobrazování surových dat posílaných z mikrokontrolérů v~robotech, digitálních sondách apod. 
Hlavní vlastností programu je modulárnost - rozdělení na~podčásti určené ke specifickým úkonům (Terminál, grafický parser, vykreslování grafů).

\D \B{Klíčová slova:} parser, analýza dat, program

\section*{Annotation}

    Purpose of this labor is to create graphical user interface for parsing and displaying raw data sent from embedded devices, robots, digital probes and other devices which are using microcontrollers.
Main feature of this application is modularity - it is divided to sub-sections designed for specific operations (Terminal, graphical parser, graph drawer).

\D \B{Key words:} parser, data analysis, program

\addtolength{\textheight}{30mm} % prodlouží následující stránku

%%%%%%%%%%%%%%%%%%%%%%%%%%%%%%%%%%%%%%%%%%%%%%%%%%%%%%%%%%%%%%%%%%%%%%%%%%%
\newpage

\setlength{\voffset}{-20mm} % posune text/obrázek na této stránce nahoru
%\setcounter{page}{1}  % nastaví čítač stránek znovu od jedné

\tableofcontents  % vysází obsah

\addtolength{\textheight}{-30mm} % zkrátí následující stránku
%%%%%%%%%%%%%%%%%%%%%%%%%%%%%%%%%%%%%%%%%%%%%%%%%%%%%%%%%%%%%%%%%%%%%%%%%%%
\newpage
\setlength{\voffset}{0mm} % posune text/obrázek na této stránce, kam patří
%\pagestyle{headings} % znovu zapne číslování
\pagestyle{plain}
\section*{Úvod}

\addcontentsline{toc}{section}{Úvod} % přidá položku úvod do obsahu
% zde začne text úvodu
%TODO: naší?
Díky fyzikálně-matematickém kroužku na~naší škole jsem se seznámil s~mikrokontroléry, které používáme k řízení robotů určených například pro soutěž Eurobot nebo jako učební pomůcka. S~tím vyvstal problém s~parsováním dat ze senzorů v~čitelnější podobě, než je běžný terminál, která by zároveň byla snadno a~rychle nastavitelná. Žádný program splňující tato kritéria jsem nenašel, a~proto jsem se začal zabývat vlastním řešením.
 
\subsection*{Stanovené cíle}

    \begin{itemize}
        \item Modulárnost - jednoduché vytváření dalších modulů pro specifické činnosti.

        \item Jednoduché a~rychlé určení formátu zdrojových dat.

        \item Absolutní kontrola nad způsobem zobrazování parsovaných dat.

        \item Využití Qt Frameworku - možnost běhu programu na různých platformách.

    \end{itemize}

\subsection*{Osobní cíle}
%TODO
    Qt Framework jsem zvolil protože umožňuje jednoduché vytvoření GUIjsem chtěl zkusit naprogramovat program s~GUI (do té doby jsem pracoval většinou se serverovým emulátorem MaNGOS \cite{mangos} a programy pro mikrokontroléry). Qt umožňuje vytvoření GUI pro vícero platforem s~jedním zdrojovým kódem.


%%%%%%%%%%%%%%%%%%%%%%%%%%%%%%%%%%%%%%%%%%%%%%%%%%%%%%%%%%%%%%%%%%%%%%%%%%%

\newpage
\addcontentsline{toc}{subsection}{Literatura}


 \begin{thebibliography}{99}
 %% 99 znamená, že maximální délka čísla literatury jsou dva znaky
% seznam samozřejmě změníte podle svého, tohle je pouze ukázka formátování

  \bibitem{mangos} \It{Massive Network Game Object Server - MaNGOS} \\
   \verb#http://getmangos.com# (Stav ke dni 5.11.2011)

\end{thebibliography}

\end{document}